\chapter{Introduction} % ~ 1-2 pages. Define the problem.
%The goal of this project is to extend the Indian Buffet Process (IBP) to include
%distance dependency. 

The goal of this project is to define a distribution over sparse binary of
infinite dimensions. Specifically, we want to extend the Indian buffet Process
to include distance information between observations. \\

\noindent
The Indian buffet process (IBP) provides a distribution for binary matrices of
infinite dimensions. They are a good choice of prior distribution for feature
matrices in infinite latexnt feature models, as their dimensions can be learned
from the data and do not have to be predetermined.\\

\noindent
Implementations of the IBP to include distance information have been studied and
introduced by authors such as \cite{ddibp}. The distance-dependent Indian buffet
process (ddIBP) is one such implementation. It reduces to the regular IBP under
certain conditions, and preserves many of the properties of the IBP. We are 
interested in proposing a new distribution which makes use of attraction information
that \cite{epa} uses in his Ewen-Pitman attraction (EPA) distribution. We
also wish to make a comparison between the ddIBP and the proposed distribution.
We hope to preserve as many features of the IBP as possible. The proposed 
distribution will also have a p.m.f which does not require the enumeration of all
possible connectivity matrices and ownership vectors, as in the ddIBP.

