\chapter{Research Conclusions \& Future Work}
The unique properties of the AIBP are that (1) it has an explicit p.m.f., and
(2) it uses distance information to tilt the distribution of dish-taking
while respecting the natural dimensions of the IBP. Having an explicit p.m.f.
is certainly more frequently a modeling advantage. It is not clear whether 
preservation of dimensions is an advantage of the AIBP over the ddIBP. More study
will be done to investigate the situations in which this property may be 
appropriate for a modeling choice.\\

\noindent
In order to make use of the p.m.f. in MCMC to obtain posterior distributions, an
appropriate proposal mechanism needs to be developed. The complexity occurs when
the dimensions of the realizations from the IBP are changed from draw to draw. 
The proposal mechanism needs to account for added columns, removed columns, and 
changed values of the matrices drawn. \\

\noindent
So far, we have not discussed a practical area of application for the IBP.
But there may be appropriate and natural applications in the development Bayesian
mixed models. The linear mixed model can be written as
\[
  \bm{y = X\beta + Z\gamma + \epsilon}.
\]
Traditionally, $\bm Z$ is a design matrix determined by researchers. The design
matrix can consist of random intercepts for each subject in the dataset, and
could also include random slopes. An interesting application of the IBP could
be to model the design matrix $\bm Z$ with an IBP prior. Though it is usually
predetermined, modeling $\bm Z$ may be a way of determining latent features (in
this case, intercepts) that are generating the observations. If a subset of the
observations are generating from common intercepts, modeling $\bm Z$ may be of
value. Adding distance information using the AIBP may be suitable if we know
that observations that are similar are more likely to share intercepts.
However, no formal research has been conducted on this subject, and the
application of the IBP in mixed models warrants further investigation.
