\chapter{Research Plan}%~3-8 pages
%  - Model.
%  - What am I going to do?
%  - Future tense
%  - 3-8 pages

We will propose a distribution for infinitely sparse matrices, incorporating
distance information. Though Frazier and Blei have defined the distance
dependent Indian buffet process (ddIBP), the distribution we are proposing will
incorporate a measure of attraction similar to in the EPA distribution. We will
perform simulations to explore and compare the properties of the proposed
distribution and the ddIBP. For instance, we will explore expected row sums and
column sums of random draws from the proposed distribution, expected value for
matrices, expected value of rows given a predetermined set of previous rows,
etc. Time permitting, we will also develop an MCMC method to make posterior
inference, and explore the theoretical properties of and applications of the
proposed distribution. \\

\noindent
The p.m.f for $\bm Z$ in the ddIBP involves the enumeration of all possible sets
of connectivity matrices and ownership vectors that generates the same $\bm Z$ 
matrix. This becomes infeasible very quickly as the dimensions of $\bm Z$ 
increases. Evaluating the p.m.f, therefore, cannot be done except for smaller
dimensions of $\bm Z$, and typical Metropolis-Hastings algorithms based just on 
the likelihood and prior, cannot be conveniently implemented. We suspect that
we can write a p.m.f. in closed form for the proposed distribution.\\

\noindent
The new distribution we will propose will incorporate attraction information to
determine how customers are connected by dishes. It will also preserve properties 
of the original IBP, such as expected number of dishes drawn by each customer,
expected number of dishes drawn in total, and expected sum of the $\bm Z$ matrix.

% What are the applications of such a distribution?
% What are the values of having such a distrubition?
%The value of having such a distribution is that we can model 
