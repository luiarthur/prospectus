% Topics: Talk about paper
% Here's what I am talking about, and pull the paper to show it's true

\chapter{Literature Review} % ~15 pages
We will first review the Indian buffet process (IBP). We will then review the 
distance dependent Indian buffet process, formulated by Gershman. % Cite
Finally, we will finally review existing McMC methods for sampling from the IBP.
% Maybe applications also?

\subsection{The Indian buffet process}
The Indian buffet process is a distribution on infinite sparse (left-ordered)
binary matrices. (i.e. matrices with finite number of rows and infinite number
of columns, ordered by the magnitude of their columns when expressed as a
binary number.) The IBP is a multivariate extension of the Chinese restaurant
process. For an N x K matrix, Z, that follows an IBP($\alpha$) distribution,
element $z_{ik}$ is 1 if observation (customer) i possess feature (dish) k, and
0 otherwise.  And $\alpha$ is a parameter that determines the sparsity of the
matrix. The larger $\alpha$ is, the more likely Z will be sparse. The process
can be generated as follows: \\

\noindent
N customers enter a buffet one after another. The buffet line contains an
infinite number of dishes. The first customer takes the first Poisson($\alpha$)
number of dishes. The $i^{th}$ customer then takes previously sampled dishes
with probability proportional to their popularity, serving himself with
probability $\frac{m_k}{i}$, where $m_k$ is the number of people that had
previously taken dish k. After customer i has sampled all the previously sampled
dishes, he samples Poisson($\frac{\alpha}{i}$) new dishes. The probability of 
any matrix being generated by this process is
\begin{equation}
  P(\bm{Z}) = \frac{\alpha^{K_+}}{\prodl{i}{1}{N} K_1^{(i)}!} 
              exp\{-\alpha H_N\}\prodl{k}{1}{K_+}
              \frac{(N-m_k)!(m_k-1)!}{N!},
\end{equation}
where $H_N$ is the harmonic number, $\suml{i}{1}{N}\ds\frac{1}{i}$, $K_+$ is the 
number of non-zero columns in $\bm Z$, and $K_1^{(i)}$ is the number of new dishes
sampled by customer i.

\subsection{Gibbs sampler}


% What is the IBP?
% Gibbs sampler to draw from prior
% Gibbs sampler to draw from posterior
% What is it used for?
% Where are the applications?
% Why would it be useful to include distance information?
% ddIBP?

